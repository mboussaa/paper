\section{Introduction}
Compiler users tend to improve software programs in a safe and profitable way. Modern compilers provide a broad collection of optimizations that can be applied during the code generation process. 
For functional testing of compilers, software testers generally use to re-execute a set of test suites on different optimized software versions and compare the functional outcome that can be either pass (correct behaviour) or fail (incorrect behaviour, crashes or bugs)~\cite{hoste2008cole,le2014compiler}.

For non-functional testing, improvement of source code programs in term of performance can refer to several different non-functional properties of the produced code such as code size, resource or energy consumption, execution time, among others~\cite{almagor2004finding,pan2006fast}.
Testing non-functional properties is more challenging because compilers may have a huge number of potential optimization combinations, making it hard and time-consuming for software developers to find/construct the sequence of optimizations that satisfies user specific key objectives and criteria. It also requires a comprehensive understanding of the underlying system architecture, the target application and the available optimizations of the compiler.

In some cases, these optimizations may negatively decrease the quality of the software and deteriorate application performance over time~\cite{molyneaux2009art}. 
As a consequence, compiler creators usually define fixed and program-independent sequence optimizations, which are based on their experiences and heuristics. For example, in GCC, we can distinguish optimization levels from O1 to O3. Each optimization level involves a fixed list of compiler optimization options and provide different trade-offs in term of non-functional properties.
Nevertheless, there is no guarantee that these optimization levels will perform well on untested architectures or for unseen applications. 
Thus, it is necessary to detect possible issues caused by source code changes such as performance regressions and help users to validate optimizations that induce to performance improvement.
%Many technologies, such as Docker containers, provide new opportunities to automate the deployment of produced code into a distributed and heterogeneous component-based infrastructure

We note also that when trying to optimize the performance of a software application,
many non-functional properties and design constraints must be involved and satisfied simultaneously to better optimize code.
Most of previous works try to optimize a single criterion (usually the execution time)~\cite{ballal2015compiler,chen2012deconstructing,demertzi2011analyzing} and ignore other important non-functional properties, more precisely resource consumption properties such as memory or CPU usage, that must be taken into consideration and can be equally important in relation to the performance. Sometimes, optimizing for execution time can lead to higher resource consumption and vice versa. For example, embedded systems for which code is generated often have limited resources. Therefore, optimization techniques must be applied whenever possible to generate efficient code and improve performance (in term of execution time) with respect to available resources (in term of CPU or memory usage)~\cite{nagiub2013automatic}.
Therefore, it is important to construct optimization levels that represent multiple trade-offs between non-functional properties, enabling the software designer to choose among different optimal solutions which best suit the system specifications.



%In this context, compilers suppose a decisive factor in the optimization. Compilers
%currently offer a large number of optimization options. However, this capability
%is never fully exploited as it involves a comprehensive understanding of
%the underlying computer architecture, the target application and the operation
%of the compiler.

 

In this paper, we propose NOTICE (as NOn-functional TestIng of CompilErs), a component-based framework for non-functional testing of compilers through the monitoring of generated code in a controlled sand-boxing environment. Our approach is based on micro-services to automate the deployment and monitoring of different variants of optimized code. NOTICE is an on-demand tool that employs mono and multi-objective evolutionary search algorithms to construct optimization sequences that satisfy user key objectives (execution time, code size, compilation time, CPU or memory usage, etc.).
In this paper, we make the following contributions:
\begin{itemize}  
	
	
	
	\item The paper introduces a novel formulation of compiler optimizations exploration problem using Novelty Search~\cite{lehman2008exploiting}. We evaluate the effectiveness of our approach by verifying the optimizations performed by GCC compiler.
	Our experimental results show that NOTICE is able to auto-tune compilers according to user choices (heuristics, objectives, programs, etc.) and construct optimizations that yield to better performance results than standard optimization levels.
	% and, to the best of our knowledge, this is the first paper in the literature to %use Novelty Search to evaluate optimization sequences.
	
	\item We propose a micro-service infrastructure to ensure the deployment and monitoring of different variants of optimized code. In this paper, we focus more on the relationship between runtime execution of optimized code and resource consumption profiles (CPU and memory usage) by providing a fine-grained understanding and analysis of compilers behavior regarding optimizations.
	

	\item We also demonstrate that NOTICE can be used to automatically construct optimization levels that represent optimal trade-offs between multiple non-functional properties, such as execution time, memory usage, CPU consumption, etc.
\end{itemize}


 
%This to ensure the efficiency of generated code, deployed components must be checked and verified regarding their non-functional behavior
%\footnote{\url{https://www.docker.com}} 



The paper is organized as follows.
Section II describes the motivation behind this work. A search-based technique for compiler optimizations exploration is presented in Section III. 
We present in Section IV our infrastructure for non-functional testing using microservices.  The evaluation and results of our experiments across two case studies are discussed in Section V. 
Finally, related work, concluding remarks and future work are provided in Sections VI and VII.


%The difference between classical compilers like GCC, LLVM and code generators is that code generators are used only by few people comparing to famous compilers like GCC. Moreover, code generators face a high rate of changes and update versions due to new development needs. Hence, it becomes necessary to check the quality of produced code. This will help software maintainers to verify the correct functioning of code generators.  Among the most important properties to check, we distinguish the non functional properties.

%Like traditional compilers, code generators typically perform multiple passes over various intermediate forms during code transformation. Many rules may also be applied and that differs from one generator to another. These passes are complex and highly dependent on target platform architecture. 

 
 

  % \section{Motivations}
 %\section{Previous work}
%\section{Approach Overview}
