\section{Related Work}
Our work is related to iterative compilation research field.
The basic idea of iterative compilation is to explore the compiler optimization space by measuring the impact of optimizations on software performance.
Several research efforts have investigated this optimization problem using search-based techniques (SBSE) to guide the search toward relevant optimizations regrading performance, energy consumption, code size, compilation time, etc. Experimental results have been usually compared to standard compiler optimization levels.  
The vast majority of the work on iterative compilation focuses on increasing the speedup of new optimized code compared to standard compiler optimization levels~\cite{almagor2004finding,hoste2008cole,pan2006fast,zhong2009tuning,pallister2015identifying,chen2012deconstructing,sandran2012genetic,martins2014exploration,fursin2008milepost,lin2008automatic,schulte2014post,martinez2014multi}.
It has been proven that optimizations are highly dependent on target platform and input program. 
Compared to our proposal, none of previous work has studied the impact of compiler optimizations on resource usage. In this work, we rather focus on compiler optimizations related to resource consumption, while bearing in mind performance improvement.

Novelty Search has never been applied in the field of iterative compilation. Our work presents the first attempt to introduce diversity in the compiler optimization problem. 
The idea of NS has been introduced by Lehman et al.~\cite{lehman2008exploiting}. It has been often evaluated in deceptive tasks and especially applied to evolutionary robotics~\cite{risi2010evolving,krvcah2012solving} (in the context of neuroevolution). 
NS can be easily adapted to different research fields. In the field of software testing, as we introduced, NS has been applied to only one work~\cite{boussaa2015novelty}. Authors in this work have adapted the general idea of NS to the test data generation problem where novelty score was calculated as the Manhattan distance between the different vectors representing test data. The evaluation metric of generated test suites is the structural coverage of code.
In our NS adaptation, the evaluation metric represents the non-functional measurements and we are measuring the novelty score using the systematic difference between optimization sequences of GCC. 

For multi-objective optimizations, we are not the first to address this problem. New approaches have emerged recently to find trade-offs between non-functional properties~\cite{hoste2008cole,martinez2014multi,lokuciejewski2010multi}. Hoste et al.~\cite{hoste2008cole}, which is the most related work to our proposal, propose COLE, an automated tool for optimization generation using a multi-objective approach namely SPEA2. In their work, they try to find Pareto optimal optimization levels that present a trade-off between execution and compilation time of generated code. Their experimental results show that the obtained optimization sequences perform better than GCC standard optimization levels. NOTICE provides also a fully automated approach to extract non-functional properties. However, NOTICE differs from COLE because first, our proposed container-based infrastructure is more generic and can be adapted to other case studies (i.e., compilers, code generators, etc.). Second, we provide facilities to compiler users to extract resource usage metrics using our monitoring components. Finally, our empirical study investigates different trade-offs compared to previous work in iterative compilation.



%For code generators testing, Stuermer et al.~\cite{stuermer2007systematic} present a systematic test approach for model-based code generators. They investigate the impact of optimization rules for model-based code generation by comparing the output of the code execution with the output of the model execution. 
%If these outputs are equivalent, it is assumed that the code generator works as expected. 
%They evaluate the effectiveness of this approach by means of testing optimizations performed by the TargetLink code generator. 
%They have used Simulink as a simulation environment of models. 
%In our approach, we provide a component-based infrastructure to compare non-functional properties of generated code rather than functional ones. 

